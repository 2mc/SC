% Created 2011-05-15 Sun 20:26
\documentclass[11pt]{article}
\usepackage[utf8]{inputenc}
\usepackage[T1]{fontenc}
\usepackage{fixltx2e}
\usepackage{graphicx}
\usepackage{longtable}
\usepackage{float}
\usepackage{wrapfig}
\usepackage{soul}
\usepackage{textcomp}
\usepackage{marvosym}
\usepackage{wasysym}
\usepackage{latexsym}
\usepackage{amssymb}
\usepackage{hyperref}
\tolerance=1000
\providecommand{\alert}[1]{\textbf{#1}}
\begin{document}



\title{Lilt2/Elemenb\'i (\emph{Look Ma, No Boot})}
\author{Ioannis Zannos}
\date{15 May 2011}
\maketitle

\setcounter{tocdepth}{3}
\tableofcontents
\vspace*{1cm}

By Ioannis Zannos, March-May 2011

Download from: \href{https://github.com/iani/SC}{https://github.com/iani/SC}
or:
\begin{verbatim}
  git clone git://github.com/iani/SC.git
\end{verbatim}


\textbf{Summary of ideas in this Library}


\section{Class ServerPrep}
\label{sec-1}


\begin{itemize}
\item Obviate the need to boot the server manually before starting synths.
\item Ensure that Buffers and SynthDefs are allocated / sent to the server
  before starting synths, efficiently.
\item Provide a safe way for registering synth and routine processes to start automatically when the server boots
  or when the tree is inited, ensuring that SynthDefs and Buffers will be loaded first.
\end{itemize}

Classes involved: 

\begin{itemize}
\item ServerPrep
\item ServerActionLoader
\item SynthLoader
\item DefLoader
\item BufLoader
\item RoutineLoader
\item UniqueBuffer
\item Udef
\end{itemize}
\section{Class UniqueSynth}
\label{sec-2}


Simplify the creation and control of Synths by storing them in a dictionary for later access, and by providing utility methods for
controlling the duration and release time, for synchronizing the execution and life time of routines pertaining to a synth, and for attaching other objects that react to the start and end of a synth.

Example of how UniqueSynth can simplify the code required: 

\emph{Without Symbol:mplay}

\begin{verbatim}
(
{
   loop {
      {    var synth;
         synth = Synth(\default, [\freq, (25..50).choose.midicps]);
         0.1.wait;
         synth.release(exprand(0.01, 1.0));
      }.fork;
      [0.1, 0.2].choose.wait;
   };
}.fork;
)
\end{verbatim}

\emph{Using Symbol:mplay}

\begin{verbatim}

(
{
   loop {
      \default.mplay([\freq, (25..50).choose.midicps])
         .dur(0.1, exprand(0.01, 1.0));
      [0.1, 0.2].choose.wait;
   };
}.fork;
)
\end{verbatim}
\section{Class Chain, EventStream, Function:sched and Function:stream}
\label{sec-3}


Simplify the creation and access of Streams from Patterns and their use with Routines and Functions scheduled for repeated execution.  

Example: Simplify the above code even further, while enabling  control of dtime (and any other parameters) via patterns:

\begin{verbatim}
(
{   // Symbol:stream creates and / or accesses the stream as appropriate: 
   \default.mplay([\freq, \freq.prand((25..50), inf).midicps])
      .dur(0.1, exprand(0.01, 1.0));
   // play 20 events only
   \duration.stream(Prand([0.1, 0.2], 20)); 
}.stream;    
)
\end{verbatim}

Note: symbol.stream(Prand(\ldots{})) is equivalent to symbol.prand(\ldots{})

Also chain timed sequential execution of functions, with sound or not, in a manner more direct than Pbind.

\begin{verbatim}
(
//:3 different synth functions sharing patterns. 
Chain(Pseq([
        { \default.play([\amp, 0.05, \freq, ~freq.next]).dur(~dur2.next, ~fade.next); },
        { { Resonz.ar(WhiteNoise.ar(2.5), \freq.n.dup, 0.01) }.play.dur(\dur2.n, \fade.n); },
        { { SinOsc.ar(\freq.n.dup / 2, 0, 0.07) }.play.dur(\dur2.n, \fade.n); },
], 20), 
() make: {      // store shared patterns in the global environment of the Chain:
        \dur2.pseq([0.1, 0.2], inf);
        \fade.pseq([0.1, 0.2, 1], inf); 
        \freq.pseq([80, 85, 87, 90, 92].midicps, inf) 
});
//: ---
)
\end{verbatim}

Other example: 

\begin{verbatim}
(
//:Example combining a single synth and a chain of synths.
Chain(Prand([ // choose from the following at random:
        {       // Play a series of events
                \default.mplay([\freq, (50..80).choose.midicps]).dur(0.03, exprand(0.01, 0.3));
                // The number and timing of the events is defined through arguments to the chain message
        }.chain({ Prand([0.06, 0.07, 0.14], 10 rrand: 20) }),
        {       // Play a single synth.
                { | freq = 400 | SinOsc.ar(freq * [1, 1.2], 0, 0.02) }
                        .play(args: [\freq,  \freq.pseries(4).next * 100])
                        .dur(0.1 rrand: 1, 0.5 rrand: 2.5) 
        }
], 30
));
//: ---
)
\end{verbatim}
\section{Object methods for easy messaging via NotificationCenter}
\label{sec-4}


Simplify the connection of objects for sending messages to each other via NotificationCenter. Automate the creation of mutual NotificationCenter registrations to messages, and their removal when an object receives the message objectClosed. This makes it easier to establish messaging between objects in the manner of the Observer pattern exemplified by classes Model and SimpleController, while shotening and clarifying the code required to use NotificationCenter.

One beneficial effect of this is that it is no longer needed to check whether an object stored in a variable is nil in order to decide whether to send it a message. One can create messaging interconnections between objects without storing one in a variable of the other, and one can safely send a message to an object before it is created or after it is no longer a valid receiver of that message. 
\section{Class Code}
\label{sec-5}


Enable the selection of parts of a SuperCollider document separated by comments followed by :, the movement between such parts, and the execution of those parts through keyboard shortcuts. Additionally, wrap these code parts in a routine so that number.wait messages can be written straight in the code, without wrapping them in \{ \}.fork or Routine(\{ \}). 

Also ensure that the code will run after the default server is booted and the Buffers and SynthDefs defined as Udefs in a Session have been loaded. 

Shortcuts provided are: 

\begin{itemize}
\item Command-shift-x: Evaluate the code in an AppClock routine, after booting the default server if needed
\item Command-shift-alt-x: Evaluate the code in a SystemClock routine, after booting the default server if needed
\item Command-shift-v: Evaluate and post the results of the code, without routine or server booting
\item Command-shift-j: Select the next code part
\item Command-shift-k: Select the previous code part
\end{itemize}
\section{Class Panes}
\label{sec-6}


Arrange Document windows on the screen conveniently for maximum view area on the screen. Provide 2 layouts: single pane and 2 panes side by side, with keyboard shortcuts for switching between them. Provide an auto-updating document list palette for selecting documents by mouse or by string search. Provide a way for switching between a dark colored document theme and the default document theme via keyboard shortcuts, with automatic updating of the coloring of all relevant documents. 
\section{Class Dock}
\label{sec-7}


Provide some useful shortcuts for common tasks: 
   browseUserClasses :    Open a list of all classes defined in the user's Application Support 
      directory. Typing return on a selected item opens the code file with the definition of this class. 

   insertClassHelpTemplate : Insert a template for documenting a class named after the name of the
      document. Inserts listings of superclasses, class and instance variables and methods. 

   openCreateHelpFile : Open a help file for a selected user class. Automatic creation of the file 
         is reserved to code residing outside the distribution files of this library. 

   showDocListWindow :  An auto-updating window listing all open Documents, with selection by mouse click
               or by text search.

   closeDocListWindow : Close the document list window
\section{Class Spectrograph}
\label{sec-8}


An example application showing some of the features of this library. Creates a window showing a live running spectrogram of one of the audio channels. The fft polling process for the spectrogram is persistent, that is, it starts as soon as the server boots and re-starts if the server's processes are killed by Command-. It (optionally) stops when the Spectrograph window is closed. 

This class was inspired by the Spectrogram Quark by Thor Magnusson and Dan Stowell, and is a rewrite to show how the code can be made clearer (and the behavior safer and more consistent regarding boot/quit of the server and open/close of the spectrogram window). 

Note: The Spectrograph may occasionally crash SuperCollider if it is running on a MacBook with battery power. I have not been able to trace the source of the problem so far but suspect this is due to fast Image updates causing problems with the Graphics Card.

\end{document}
